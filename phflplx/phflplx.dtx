% \iffalse meta-comment
%
% Copyright (C) 2021 by Philippe Faist, philippe.faist@bluewin.ch
% -------------------------------------------------------
% 
% This file may be distributed and/or modified under the
% conditions of the LaTeX Project Public License, either version 1.3
% of this license or (at your option) any later version.
% The latest version of this license is in:
%
%    http://www.latex-project.org/lppl.txt
%
% and version 1.3 or later is part of all distributions of LaTeX 
% version 2005/12/01 or later.
%
% \fi
%
% \iffalse
%<*driver>
\ProvidesFile{phflplx.dtx}
%</driver>
%<package>\NeedsTeXFormat{LaTeX2e}[2005/12/01]
%<package>\ProvidesPackage{phflplx}
%<*package>
    [2021/04/09 v0.1 phflplx package]
%</package>
%
%<*driver>
\documentclass{ltxdoc}
\usepackage{xcolor}
\usepackage{lipsum}
\usepackage[preset=xpkgdoc]{phfnote}
\usepackage{phflplx}

\usepackage{needspace}

\def\eqsign@{=}
\def\eqsign{\protect\eqsign@}
\robustify\eqsign
\makeatother

\EnableCrossrefs
\CodelineIndex
\RecordChanges

\begin{document}
  \DocInput{phflplx.dtx}
\end{document}
%</driver>
% \fi
%
% \CheckSum{0}
%
% \CharacterTable
%  {Upper-case    \A\B\C\D\E\F\G\H\I\J\K\L\M\N\O\P\Q\R\S\T\U\V\W\X\Y\Z
%   Lower-case    \a\b\c\d\e\f\g\h\i\j\k\l\m\n\o\p\q\r\s\t\u\v\w\x\y\z
%   Digits        \0\1\2\3\4\5\6\7\8\9
%   Exclamation   \!     Double quote  \"     Hash (number) \#
%   Dollar        \$     Percent       \%     Ampersand     \&
%   Acute accent  \'     Left paren    \(     Right paren   \)
%   Asterisk      \*     Plus          \+     Comma         \,
%   Minus         \-     Point         \.     Solidus       \/
%   Colon         \:     Semicolon     \;     Less than     \<
%   Equals        \=     Greater than  \>     Question mark \?
%   Commercial at \@     Left bracket  \[     Backslash     \\
%   Right bracket \]     Circumflex    \^     Underscore    \_
%   Grave accent  \`     Left brace    \{     Vertical bar  \|
%   Right brace   \}     Tilde         \~}
%
%
% \changes{v0.1}{2021/04/09}{Initial version}
%
% \GetFileInfo{phflplx.dtx}
%
% \iffalse Bypass indexing for following commands: \fi
% \DoNotIndex{\newcommand,\newenvironment,\renewcommand,\long,\def,\edef,\gdef,\xdef,\if,\else,\fi,\par,\relax,\vspace,\vskip,\hspace,\hskip,\vbox,\hbox}
% 
% \title{The \pkgname{phflplx} package --- include PDF graphics with hyperlinks}
% \author{Philippe Faist\quad\email{philippe.faist@bluewin.ch}}
% \date{\pkgfmtdate\filedate}
% \maketitle
%
% \begin{abstract}
%   \pkgname{phflplx}---A handy \LaTeX{} package for including graphics defined
%   via \LaTeX\ source code files.  Designed for use with the |ltxpdflinks| tool
%   in order to include PDF graphics in documents with external or internal
%   hyperlinks.
% \end{abstract}
%
% \phantomsection\label{sec:toc}
% \inlinetoc
%
% \section{Introduction}
%
% Usage:
%
% \begin{verbatim}
% \usepackage{phflplx}
% \DeclareGraphicsExtensions{.lplx,.pdf}
% \end{verbatim}
%
% This package is designed to be used in conjunction with the |ltxpdflinks|
% command-line utility to extract PDF links into |.lplx| files.
%
% Make sure to process your PDF files before you compile your document:
% \begin{verbatim}
% > ltxpdflinks myfigure.pdf
% \end{verbatim}
% 
% In fact, LPLX files are simple LaTeX sources that are included directly by the
% graphics driver.  The LPLX file typically includes the PDF itself using a
% lower-level graphics command and draws invisible hyperlinks on top of the
% graphics.  (The LPLX file also takes care of resizing and cropping the
% graphics if necessary.)
%
%
% \StopEventually{\PrintChangesAndIndex}
%
% \section{Implementation}
%
% Load some useful packages.
%    \begin{macrocode}
\RequirePackage{etoolbox}
%    \end{macrocode}
%
% Check that the user is loading the \pkgname{hyperref} package! (We don't load
% it automatically to avoid issues of package loading order, link appearance,
% etc.)
%    \begin{macrocode}
\AtBeginDocument{%
  \@ifpackageloaded{hyperref}{}{%
    \PackageWarning{phflplx}{The package `hyperref' was not loaded. You
      probably forgot to load it.}%
  }%
}
%    \end{macrocode}
%
%
% Declaring the graphics rule for the LPLX extension.  Note that the LPLX file
% has the line |%%BoundingBox 0 0 <W> <H>| so that the size of the graphics can
% be parsed by the default \pkgname{graphics} graphic size inspector, which is
% designed to parse |%%BoundingBox| commands in EPS files.
%    \begin{macrocode}
\DeclareGraphicsRule{.lplx}{lplx}{*}{}
%    \end{macrocode}
% 
% Declare the driver functions for the \pkgname{graphics} package internals.
%    \begin{macrocode}
\def\Ginclude@lplx#1{%
  \message{<#1>}%
  \input{#1}%
}
%    \end{macrocode}
%
% This command gets called from within the LPLX files to actually include the
% underlying graphics.
%    \begin{macrocode}
\def\lplxIncludeGraphics{%
  \edef\x{\noexpand\hbox to 0pt{%
      \noexpand\Ginclude@graphics{\Gin@base\lplxiGraphicExt}}}%
  \x
}
%    \end{macrocode}
% 
% Declare some simple helpers that are used in our automatically generated LPLX
% files.
%    \begin{macrocode}
\newdimen\lplxiBboxWcrdim
\newdimen\lplxiBboxHcrdim
\def\lplxSetBbox#1#2#3#4{%
  \def\lplxiBboxX{#1}%
  \def\lplxiBboxY{#2}%
  \def\lplxiBboxW{#3}%
  \def\lplxiBboxH{#4}%
  \ifnum\lplxiBboxX>0\relax\lplx@bboxzerowarn\fi
  \ifnum\lplxiBboxX<0\relax\lplx@bboxzerowarn\fi
  \ifnum\lplxiBboxY>0\relax\lplx@bboxzerowarn\fi
  \ifnum\lplxiBboxY<0\relax\lplx@bboxzerowarn\fi
}
\def\lplx@bboxzerowarn{%
  \PackageWarning{phflplx}{LPLX bounding box is not pinned at (0,0), not supported}%
}
%    \end{macrocode}
% 
% Utility to set up the appropriate command arguments to use for |\scalebox|,
% etc.
%    \begin{macrocode}
\def\lplx@noscale{1,!}
\def\lplx@exclam{!}
\def\lplxSetupScaleAndBbox{%
%    \end{macrocode}
% 
% First, we locally define a macro |\lplxaDoScale{...}| that will generate the
% correct |\scalebox| call with the given contents, according to the requested
% size.
%    \begin{macrocode}
  \def\lplx@tmp{\Gin@scalex,\Gin@scaley}%
  \ifx\lplx@tmp\lplx@noscale%
    \def\lplxaDoScale{}%
  \else
    \ifx\Gin@scaley\lplx@exclam
      \edef\lplxaDoScale{\noexpand\scalebox{\Gin@scalex}}%
    \else
      \ifx\Gin@scalex\lplx@exclam
        \edef\lplxaDoScale{\noexpand\scalebox{\Gin@scaley}}%
      \else
        \edef\lplxaDoScale{\noexpand\scalebox{\Gin@scalex}[\Gin@scaley]}%
      \fi
    \fi
  \fi
%    \end{macrocode}
% Second, we need to take care of setting the bounding box correctly.  Define
% |\lplxvCropX| and |\lplxvCropY| which are the $(X,Y)$ position of the lower
% left corner the part of the image we want to pick out, in user space units.
% Define |\lplxvCropW| and |\lplxvCropH| as the requested width \& height of the
% subimage we want to use.
%    \begin{macrocode}
  \edef\lplxvCropX{\Gin@llx}%
  \edef\lplxvCropY{\Gin@lly}%
  \edef\lplxvCropW{\strip@pt\dimexpr\Gin@urx pt-\Gin@llx pt\relax}%
  \edef\lplxvCropH{\strip@pt\dimexpr\Gin@ury pt-\Gin@lly pt\relax}%
}
%    \end{macrocode}
%
% Finally, a utility that will help us place links in the picture environment.
% Usage is
% |\lplxPutLink|\marg{x}\marg{y}\marg{w}\marg{h}\marg{hyperstartcmd}\marg{hyperendtokens}.
% Here \marg{hyperstartcmd} and \marg{hyperendtokens} are any tokens that will
% be inserted immediately before and immediately after an invisible rule of the
% given width and height.  The rule will be in curly braces, so it can be
% considered as a mandatory argument to the last macro token in
% \marg{hyperstartcmd}.
%    \begin{macrocode}
\def\lplxPutLink{%
  \ifGin@clip
    \expandafter\lplx@clipputlink
  \else
    \expandafter\lplx@doputlink
  \fi
}%
\def\lplx@doputlink#1#2#3#4#5#6{% x,y,w,h,hyperstart,hyperend
  \put(#1,#2){#5{\phantom{\rule{#3bp}{#4bp}}}#6}%
}%
\newdimen\lplx@tmpx
\newdimen\lplx@tmpy
\newdimen\lplx@tmpw
\newdimen\lplx@tmph
\def\lplx@clipputlink#1#2#3#4#5#6{% x,y,w,h,hyperstart,hyperend
%    \end{macrocode}
% Some notes: 1) We use ``pt'' as dummy unit of measure here just to do the
% floating point arithmetic and we use |\strip@pt| at the end.  2) Here
% |\lplx@maybeskip| serves as a flag that if set, asserts the link was entirely
% cropped out of the picture.  Initially it expands to an empty string but when
% set it expands to |\p@<\z@\relax| (=``|1pt < 0pt|''), so it can be placed in
% front of all |\ifdim|'s so that they are skipped if the link was determined to
% be out of the picture.
%    \begin{macrocode}
  \def\lplx@maybeskip{}%
  \def\lplx@setskip{\def\lplx@maybeskip{\p@<\z@\relax}}%
  \lplx@tmpx=#1pt\relax
  \lplx@tmpy=#2pt\relax
  \lplx@tmpw=#3pt\relax
  \lplx@tmph=#4pt\relax
  \ifdim\lplx@maybeskip\lplx@tmpx<\lplxvCropX\p@\relax
    \ifdim\dimexpr\lplx@tmpx+\lplx@tmpw>\lplxvCropX\p@\relax
      \lplx@tmpw=\dimexpr\lplx@tmpx+\lplx@tmpw-\lplxvCropX\p@\relax
      \lplx@tmpx=\lplxvCropX\p@\relax
    \else
      \lplx@setskip
    \fi
  \fi
  \ifdim\lplx@maybeskip\dimexpr\lplx@tmpx+\lplx@tmpw>\dimexpr\lplxvCropX\p@+\lplxvCropW\p@\relax
    \ifdim\lplx@tmpx<\dimexpr\lplxvCropX\p@+\lplxvCropW\p@\relax
      \lplx@tmpw=\dimexpr\lplxvCropX\p@+\lplxvCropW\p@-\lplx@tmpx\relax
    \else
      \lplx@setskip
    \fi
  \fi
  \ifdim\lplx@maybeskip\lplx@tmpy<\lplxvCropY\p@\relax
    \ifdim\dimexpr\lplx@tmpy+\lplx@tmph>\lplxvCropY\p@\relax
      \lplx@tmph=\dimexpr\lplx@tmpy+\lplx@tmph-\lplxvCropY\p@\relax
      \lplx@tmpy=\lplxvCropY\p@\relax
    \else
      \lplx@setskip
    \fi
  \fi
  \ifdim\lplx@maybeskip\dimexpr\lplx@tmpy+\lplx@tmph>\dimexpr\lplxvCropY\p@+\lplxvCropH\p@\relax
    \ifdim\lplx@tmpy<\dimexpr\lplxvCropY\p@+\lplxvCropH\p@\relax
      \lplx@tmph=\dimexpr\lplxvCropY\p@+\lplxvCropH\p@-\lplx@tmpy\relax
    \else
      \lplx@setskip
    \fi
  \fi
  \ifdim\lplx@maybeskip\p@>\z@\relax
    \edef\x{\noexpand\lplx@doputlink%
      {\strip@pt\lplx@tmpx}{\strip@pt\lplx@tmpy}{\strip@pt\lplx@tmpw}{\strip@pt\lplx@tmph}}%
    \message{*****\detokenize\expandafter{\x}*******}%
    \x{#5}{#6}%
  \fi
}%

%    \end{macrocode}
% 
%\Finale
\endinput
